\documentclass[twoside]{article} 
\usepackage[utf8]{inputenc}
\usepackage{amsmath}
\usepackage{amsfonts}
\usepackage[pdftex]{graphicx}
\usepackage[procnames]{listings}
\usepackage{color}
\usepackage{lipsum} % Package to generate dummy text throughout this template
\usepackage{braket}
\usepackage{epsfig}
\usepackage{epstopdf}


% \usepackage[sc]{mathpazo} % Use the Palatino font
\usepackage[T1]{fontenc} % Use 8-bit encoding that has 256 glyphs
% \linespread{1.05} % Line spacing - Palatino needs more space between lines
% \usepackage{microtype} % Slightly tweak font spacing for aesthetics

\usepackage[hmarginratio=1:1,top=32mm,columnsep=20pt]{geometry} % Document margins
\usepackage{multicol} % Used for the two-column layout of the document
\usepackage[width=.8\textwidth,hang, small,labelfont=bf,up,textfont=it,up]{caption} % Custom captions under/above floats in tables or figures
\usepackage{float} % Required for tables and figures in the multi-column environment - they need to be placed in specific locations with the [H] (e.g. \begin{table}[H])
\usepackage{hyperref} % For hyperlinks in the PDF

\usepackage{lettrine} % The lettrine is the first enlarged letter at the beginning of the text
\usepackage{paralist} % Used for the compactitem environment which makes bullet points with less space between them
\usepackage{todonotes}
\newcommand{\unit}[1]{\ensuremath{\; \mathrm{#1}}}

\usepackage[super]{nth}
\usepackage{bm}
\usepackage[version=4]{mhchem}
\usepackage[inline]{enumitem}

\usepackage{abstract} % Allows abstract customization
\renewcommand{\abstractnamefont}{\normalfont\bfseries} % Set the "Abstract" text to bold
\renewcommand{\abstracttextfont}{\normalfont\small\itshape} % Set the abstract itself to small italic text

\usepackage{titlesec} % Allows customization of titles

\usepackage{fancyhdr} % Headers and footers

\usepackage[labelformat=simple]{subcaption}
\renewcommand\thesubfigure{(\alph{subfigure})}

\newcommand{\spfield}{\mathcal{E}_\ell^{(1)}}
\newcommand{\sprabi}{\Omega_1^{(1)}}
\newcommand{\me}{\mathrm{e}}
\usepackage{siunitx}
\sisetup{
	separate-uncertainty = true,
	multi-part-units = single
}

\DeclareSIUnit{\belm}{Bm}
\DeclareSIUnit{\gauss}{G}

\newcommand{\qop}[1]{\hat{#1}}
\newcommand{\qproj}[1]{\ket{#1}\bra{#1}}
\newcommand{\device}[1]{\textit{#1}}
\newcommand{\expval}[1]{\left\langle #1 \right\rangle}
\newcommand{\subcapref}[1]{\subcap{\subref{#1}}}
\newcommand{\subcap}[1]{\textbf{#1}~|}
\newcommand{\realpart}{\operatorname{Re}}

\newcommand{\matr}[1]{\bm{#1}}
\newcommand{\sx}{\matr{\sigma}_x}
\newcommand{\sy}{\matr{\sigma}_y}
\newcommand{\sz}{\matr{\sigma}_z}
\newcommand{\eff}{\text{eff}}
\newcommand{\real}{\text{real}}

\DeclareCaptionLabelSeparator{pipe}{ | }

\captionsetup{% use subfigure to confine changes to subcaptions
  font={sf},
  textfont={up},
  labelsep=pipe,
  format=plain,
  width=\textwidth
}

\captionsetup[subfigure]{
	format=hang,
	labelsep=pipe
}

\usepackage[backend=biber, sorting=none, style=nature]{biblatex}
\DeclareNameAlias{sortname}{given-family}
\DeclareNameAlias{default}{given-family}
\bibliography{master}

\setlength{\marginparwidth}{3cm}

\renewcommand\thesection{\arabic{section}} % Roman numerals for the sections
\renewcommand\thesubsection{\thesection.\arabic{subsection}} % Roman numerals for subsections
\titleformat{\section}[block]{\large\scshape\centering}{\thesection.}{1em}{} % Change the look of the section titles
\titleformat{\subsection}[block]{\large}{\thesubsection.}{1em}{} % Change the look of the section titles


\pagestyle{fancy} % All pages have headers and footers
\fancyhf{}
\fancyhead[C]{Notes on project in DiVincenzo group $\bullet$ 2017} % Custom header text
\fancyfoot[RO,LE]{\thepage} % Custom footer text
\fancyfoot[CO,CE]{Jesse Slim}
\fancypagestyle{firststyle}
{	
	\fancyhf{}
	\renewcommand{\headrulewidth}{0pt}
	\fancyfoot[RO,LE]{\thepage}
	\fancyfoot[CO,CE]{Jesse Slim}
}


%----------------------------------------------------------------------------------------
%	TITLE SECTION
%----------------------------------------------------------------------------------------

\title{\vspace{-15mm}
	\fontsize{12pt}{10pt}\selectfont Research internship \\[1em]
	\fontsize{18pt}{10pt}\selectfont \textbf{Notes on project in DiVincenzo group}
} % Article title

\author{
	\large
	\textsc{Jesse Slim}\\ % Your name
}
\date{\normalsize\today\vspace{-8mm}}

%----------------------------------------------------------------------------------------


\begin{document}

\definecolor{keywords}{RGB}{255,0,90}
\definecolor{comments}{RGB}{0,0,113}
\definecolor{red}{RGB}{160,0,0}
\definecolor{green}{RGB}{0,150,0}

\lstset{language=Python, 
	basicstyle=\ttfamily\small, 
	keywordstyle=\color{keywords},
	commentstyle=\color{comments},
	stringstyle=\color{red},
	showstringspaces=false,
	identifierstyle=\color{green},
	procnamekeys={def,class}}

\maketitle % Insert title
\thispagestyle{firststyle} % Only footer on first page

%----------------------------------------------------------------------------------------
%	ABSTRACT
%----------------------------------------------------------------------------------------

\vspace{1em}
\begin{abstract}
\noindent  
Magnus expansion
	
\end{abstract}
\newpage

\tableofcontents





%----------------------------------------------------------------------------------------
%	ARTICLE CONTENTS
%----------------------------------------------------------------------------------------
\newpage
\section{The truncated Magnus expansion}
The Magnus expansion\supercite{Magnusexponentialsolutiondifferential1954,WaughAverageHamiltonianTheory2007,BlanesMagnusexpansionits2009,BlanespedagogicalapproachMagnus2010} is defined as
\begin{alignat}{2}
	&\Omega(t,t_0) &&= \sum_{n=1}^\infty \Omega_n(t,t_0),\\
	&\Omega_1(t,t_0) &&= \int_{t_0}^t \mathrm{d}t_1 \tilde{H}(t_1),\label{eq:omg-1}\\
	&\Omega_2(t,t_0) &&= -\frac{1}{2}\int_{t_0}^t \mathrm{d}t_2 \left[ \Omega_1(t_2,t_0), \tilde{H}(t_2) \right] = \frac{1}{2}\int_{t_0}^t \mathrm{d}t_1 \int_{t_0}^{t_1} \mathrm{d}t_2 \left[ \tilde{H}(t_1), \tilde{H}(t_2) \right].
\end{alignat}
We use this series to approximate the trajectory of a two-level qubit driven by an oscillating signal proportional to $\matr{\sigma}_x$ in the lab frame. To this end, we use the following rotating frame Hamiltonian:
\begin{align}
	\label{eq:H-drive}
	\matr{H}(t) = \frac{E(t)}{4}\left( \sx + \cos(2 \omega t) \sx - \sin(2 \omega t) \sy \right),
\end{align}
where we set the detuning $\Delta$ and phase offset $\phi$ of the drive both to zero.

For now, we restrict ourselves to the first order Magnus term, which is equivalent to the rotating wave approximation for constant drive. We begin by studying the case of linear drive, given by
\begin{align}
	E(t) = E_0 + E_1 t.
\end{align}
We then integrate equation (\ref{eq:omg-1}) over a full period $t_c = \pi / \omega$ of the drive Hamiltonian \ref{eq:H-drive}, starting from $t_0$:
\begin{align*}
	\Omega_1(t,t_0) = &\int_{t_0}^{t_0+t_c} \frac{E(t)}{4}\left( \sx + \cos(2 \omega t) \sx - \sin(2 \omega t) \sy \right) \, \mathrm{d}t \\
	= &\int_{t_0}^{t_0+t_c} \frac{E(t)}{4} \sx \, \mathrm{d}t + \int_{t_0}^{t_0+t_c} \frac{E(t)}{4} (\cos(2 \omega t) \sx - \sin(2 \omega t) \sy) \, \mathrm{d}t\\
	= &\frac{E_0 t_c  + E_1 (t_0 t_c + t_c^2 / 2)}{4} \sx + \frac{E_1 t_c \sin(2 \omega t_0)}{8\omega} \sx + \frac{E_1 t_c \cos(2 \omega t_0)}{8\omega} \sy \\
	= &t_c\frac{E_0 + E_1 t_0}{4} \sx + t_c^2 \frac{E_1}{8} \sx + \frac{t_c}{\omega} \left( \frac{E_1 \sin(2 \omega t_0)}{8} \sx + \frac{E_1 \cos(2 \omega t_0)}{8} \sy \right) \\
	= &t_c\left[ \frac{E_0 + E_1 t_0}{4} \sx + \frac{\pi E_1}{8 \omega} \left( \sx + \sin(2 \omega t_0) \sx + \cos(2 \omega t_0) \sy \right) \right]
\end{align*}

\printbibliography
	
\end{document}