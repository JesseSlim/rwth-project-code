\documentclass[twoside]{article} 
\usepackage[utf8]{inputenc}
\usepackage{amsmath}
\usepackage{amsfonts}
\usepackage[pdftex]{graphicx}
\usepackage[procnames]{listings}
\usepackage{color}
\usepackage{lipsum} % Package to generate dummy text throughout this template
\usepackage{braket}
\usepackage{epsfig}
\usepackage{epstopdf}


% \usepackage[sc]{mathpazo} % Use the Palatino font
\usepackage[T1]{fontenc} % Use 8-bit encoding that has 256 glyphs
% \linespread{1.05} % Line spacing - Palatino needs more space between lines
% \usepackage{microtype} % Slightly tweak font spacing for aesthetics

\usepackage[hmarginratio=1:1,top=32mm,columnsep=20pt]{geometry} % Document margins
\usepackage{multicol} % Used for the two-column layout of the document
\usepackage[width=.8\textwidth,hang, small,labelfont=bf,up,textfont=it,up]{caption} % Custom captions under/above floats in tables or figures
\usepackage{float} % Required for tables and figures in the multi-column environment - they need to be placed in specific locations with the [H] (e.g. \begin{table}[H])
\usepackage{hyperref} % For hyperlinks in the PDF

\usepackage{lettrine} % The lettrine is the first enlarged letter at the beginning of the text
\usepackage{paralist} % Used for the compactitem environment which makes bullet points with less space between them
\usepackage{todonotes}
\newcommand{\unit}[1]{\ensuremath{\; \mathrm{#1}}}

\usepackage[super]{nth}
\usepackage{bm}
\usepackage[version=4]{mhchem}
\usepackage[inline]{enumitem}

\usepackage{abstract} % Allows abstract customization
\renewcommand{\abstractnamefont}{\normalfont\bfseries} % Set the "Abstract" text to bold
\renewcommand{\abstracttextfont}{\normalfont\small\itshape} % Set the abstract itself to small italic text

\usepackage{titlesec} % Allows customization of titles

\usepackage{fancyhdr} % Headers and footers

\usepackage[labelformat=simple]{subcaption}
\renewcommand\thesubfigure{(\alph{subfigure})}

\newcommand{\spfield}{\mathcal{E}_\ell^{(1)}}
\newcommand{\sprabi}{\Omega_1^{(1)}}
\newcommand{\me}{\mathrm{e}}
\usepackage{siunitx}
\sisetup{
	separate-uncertainty = true,
	multi-part-units = single
}

\DeclareSIUnit{\belm}{Bm}
\DeclareSIUnit{\gauss}{G}

\newcommand{\qop}[1]{\hat{#1}}
\newcommand{\qproj}[1]{\ket{#1}\bra{#1}}
\newcommand{\device}[1]{\textit{#1}}
\newcommand{\expval}[1]{\left\langle #1 \right\rangle}
\newcommand{\subcapref}[1]{\subcap{\subref{#1}}}
\newcommand{\subcap}[1]{\textbf{#1}~|}
\newcommand{\realpart}{\operatorname{Re}}

\newcommand{\matr}[1]{\bm{#1}}
\newcommand{\sx}{\matr{\sigma}_x}
\newcommand{\sy}{\matr{\sigma}_y}
\newcommand{\sz}{\matr{\sigma}_z}
\newcommand{\eff}{\text{eff}}
\newcommand{\real}{\text{real}}

\DeclareCaptionLabelSeparator{pipe}{ | }

\captionsetup{% use subfigure to confine changes to subcaptions
  font={sf},
  textfont={up},
  labelsep=pipe,
  format=plain,
  width=\textwidth
}

\captionsetup[subfigure]{
	format=hang,
	labelsep=pipe
}

\usepackage[backend=biber, sorting=none, style=nature]{biblatex}
\DeclareNameAlias{sortname}{given-family}
\DeclareNameAlias{default}{given-family}
\bibliography{master}